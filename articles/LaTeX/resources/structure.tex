%----------------------------------------------------------------------------------------
%	PACKAGES AND OTHER DOCUMENT CONFIGURATIONS
%----------------------------------------------------------------------------------------

\usepackage[english]{babel} % English language hyphenation

\usepackage{flushend} % make columns in last page even length

\usepackage{microtype} % Better typography

\usepackage{amsmath,amsfonts,amsthm} % Math packages for equations

\usepackage[svgnames]{xcolor} % Enabling colors by their 'svgnames'

\usepackage[hang, small, labelfont=bf, up, textfont=it]{caption} % Custom captions under/above tables and figures

\usepackage{booktabs} % Horizontal rules in tables

\usepackage{lastpage} % Used to determine the number of pages in the document (for "Page X of Total")

\usepackage{graphicx} % Required for adding images

\usepackage{enumitem} % Required for customising lists
\setlist{noitemsep} % Remove spacing between bullet/numbered list elements

\usepackage{sectsty} % Enables custom section titles
\allsectionsfont{\usefont{OT1}{phv}{b}{n}} % Change the font of all section commands (Helvetica)

%----------------------------------------------------------------------------------------
%	MARGINS AND SPACING
%----------------------------------------------------------------------------------------

\usepackage{geometry} % Required for adjusting page dimensions

\geometry{
	top=1cm, % Top margin
	bottom=1.5cm, % Bottom margin
	left=2cm, % Left margin
	right=2cm, % Right margin
	includehead, % Include space for a header
	includefoot, % Include space for a footer
	%showframe, % Uncomment to show how the type block is set on the page
}

\setlength{\columnsep}{7mm} % Column separation width

%----------------------------------------------------------------------------------------
%	FONTS
%----------------------------------------------------------------------------------------

\usepackage[T1]{fontenc} % Output font encoding for international characters
\usepackage[utf8]{inputenc} % Required for inputting international characters

\usepackage{XCharter} % Use the XCharter font

%----------------------------------------------------------------------------------------
%	HEADERS AND FOOTERS
%----------------------------------------------------------------------------------------

\usepackage{fancyhdr} % Needed to define custom headers/footers
\pagestyle{fancy} % Enables the custom headers/footers

\renewcommand{\headrulewidth}{0.0pt} % No header rule
\renewcommand{\footrulewidth}{0.0pt} % Thin footer rule

\renewcommand{\sectionmark}[1]{\markboth{#1}{}} % Removes the section number from the header when \leftmark is used

%\nouppercase\leftmark % Add this to one of the lines below if you want a section title in the header/footer

\fancyhf {} % clear all headers and footers

%Header includes the page background
\fancyhead{
	\begin{tikzpicture}[remember picture,overlay]
		\node[inner sep=0pt] at (current page.center) {\includegraphics[width=\paperwidth,height=\paperheight]{../images/paper-white}};
	\end{tikzpicture}
}

% Left-even page footer
\fancyfoot[LE]{%
	\begin{tikzpicture}[remember picture,overlay]
		\node[xscale=-1,inner sep=0pt,anchor=south,nearly opaque] at (current page.south) {\includegraphics[width=\paperwidth,height=.6in]{../images/footerscroll.png}};
		\node[inner sep=0pt,anchor=south,xshift=.28in,yshift=.39in] at (current page.south west) {\thepage};
	\end{tikzpicture}
}

% Right-odd page footer
\fancyfoot[RO]{%
	\begin{tikzpicture}[remember picture,overlay]
		\node[inner sep=0pt,anchor=south,nearly opaque] at (current page.south) {\includegraphics[width=\paperwidth,height=.6in]{../images/footerscroll.png}};
		\node[inner sep=0pt,anchor=south,xshift=-.28in,yshift=.39in] at (current page.south east) {\thepage};
	\end{tikzpicture}
}

%----------------------------------------------------------------------------------------
%	TITLE SECTION
%----------------------------------------------------------------------------------------

\newcommand{\authorstyle}[1]{{\large\usefont{OT1}{phv}{b}{n}\color{DarkBlue}#1}} % Authors style (Helvetica)

\newcommand{\institution}[1]{{\footnotesize\usefont{OT1}{phv}{m}{sl}\color{Black}#1}} % Institutions style (Helvetica)

\usepackage{titling} % Allows custom title configuration

\newcommand{\HorRule}{\color{DarkBlue}\rule{\linewidth}{1pt}} % Defines the gold horizontal rule around the title

\pretitle{
	\vspace{-50pt} % Move the entire title section up
	\includegraphics[width=\linewidth]{../images/banner/PRP-banner.png}
	%\HorRule\vspace{10pt} % Horizontal rule before the title
	\fontsize{32}{36}\usefont{OT1}{phv}{b}{n}\selectfont % Helvetica
	\color{DarkBlue} % Text colour for the title and author(s)
}


\posttitle{\par\vskip 3pt} % Whitespace under the title

\preauthor{} % Anything that will appear before \author is printed

\usepackage{multicol, etoolbox}
\setcounter{tocdepth}{3} %set depth of printed table of contets.
\makeatletter
\patchcmd{\l@section}
{\hfil}
{\leaders\hbox{\normalfont$\m@th\mkern \@dotsep mu\hbox{.}\mkern \@dotsep     mu$}\hfill}
{}{}

\renewcommand\tableofcontents{%
	\begin{multicols}{2}
		\@starttoc{toc}%
	\end{multicols}%
}
\makeatother %print dots in sections in toc.

\postauthor{ % Anything that will appear after \author is printed
	\vspace{3pt} % Space before the rule
	\begin{center}
		\fontsize{16}{18}\usefont{OT1}{phv}{b}{n}\selectfont % Helvetica
		\color{DarkBlue} % Text colour for the title and author(s)
		Issue 001 - \today
	\end{center}
	\vspace{3pt} % Space before the rule
	\par\HorRule % Horizontal rule after the title
	\vspace{5pt} % Space after the title section
	\vspace{5pt} % Space after the title section
	\begin{minipage}{\linewidth}
		\renewcommand\contentsname{} % the empty name for toc
		\begingroup
		\let\clearpage\relax
		\vspace{0.5cm}
		\small\tableofcontents
		\thispagestyle{fancy}
		\endgroup
	\end{minipage}
}

%----------------------------------------------------------------------------------------
%	ABSTRACT
%----------------------------------------------------------------------------------------

\usepackage{lettrine} % Package to accentuate the first letter of the text (lettrine)
\usepackage{fix-cm}	% Fixes the height of the lettrine

\newcommand{\initial}[1]{ % Defines the command and style for the lettrine
	\lettrine[lines=2,findent=4pt,nindent=0pt]{% Lettrine takes up 3 lines, the text to the right of it is indented 4pt and further indenting of lines 2+ is stopped
		\color{DarkBlue}% Lettrine colour
		{#1}% The letter
	}{}%
}

\usepackage{xstring} % Required for string manipulation
	
\usepackage{stfloats} % Allows the H option in figures.

\newcommand{\abstract}[1]{
	\begin{figure*}[t]
		\centering
		\begin{minipage}{0.9\textwidth}
			\HorRule\color{black}\vspace{-6pt}
			\StrLeft{#1}{1}[\firstletter] % Capture the first letter of the abstract for the lettrine
			\initial{\firstletter}\textbf{\StrGobbleLeft{#1}{1}} % Print the abstract with the first letter as a lettrine and the rest in bold
			\HorRule
		\end{minipage}
	\end{figure*}
}

%----------------------------------------------------------------------------------------
%	Section Titles
%----------------------------------------------------------------------------------------

\setcounter{secnumdepth}{0}

\usepackage{titlesec}
\titleformat{\section}
{\normalfont\Large\bfseries}{\thesection}{1em}{}[{\titlerule[0.8pt]}]

%----------------------------------------------------------------------------------------
%	BIBLIOGRAPHY
%----------------------------------------------------------------------------------------

\usepackage[backend=bibtex,style=authoryear,natbib=true]{biblatex} % Use the bibtex backend with the authoryear citation style (which resembles APA)

\addbibresource{../resources/references.bib} % The filename of the bibliography

\usepackage[autostyle=true]{csquotes} % Required to generate language-dependent quotes in the bibliography

%----------------------------------------------------------------------------------------
%   FANCY BOXES
%----------------------------------------------------------------------------------------

\usepackage{tikz}
\usetikzlibrary{shapes,snakes}
\usepackage{amsmath, amssymb}
\usepackage{polynom}

\usepackage{wrapfig}

\usepackage{lipsum}

% Define box and box title style
\tikzstyle{mybox} = [draw=blue, fill=cyan!20, very thick, rectangle, rounded corners, inner sep=15pt]
\tikzstyle{fancytitle} =[fill=LightBlue, text=black, draw=blue, very thick]

\newcommand{\bluebox}[3]{
	\begin{wrapfigure}[#3]{r}[\dimexpr\columnwidth+\columnsep\relax]{11.5cm}
		\resizebox{12cm}{\height}{
			\begin{tikzpicture}
				\node [mybox] (box){%
					\begin{minipage}{0.70\textwidth}
						#2
					\end{minipage}
				};
				\node[fancytitle, right=10pt] at (box.north west) {\hspace{2pt}#1\hspace{2pt}};
				%\node[fancytitle, rounded corners] at (box.east) {$\clubsuit$};
			\end{tikzpicture}%
		}
	\end{wrapfigure}
}

\usepackage[framemethod=TikZ]{mdframed}
	
\newenvironment{fancybox}[2][]{%
	\ifstrempty{#1}%
	{\mdfsetup{%
			frametitle={%
				\tikz[baseline=(current bounding box.east),outer sep=0pt]
				\node[anchor=east,rectangle,fill=LightBlue]
				{\strut};}}
	}%
	{\mdfsetup{%
			frametitle={%
				\tikz[baseline=(current bounding box.east),outer sep=1pt]
				\node[draw=blue,anchor=east,rectangle,fill=LightBlue,rounded corners]
				{\hspace{3pt}#1\hspace{3pt}};}}%
	}%
	\mdfsetup{innertopmargin=5pt,linecolor=DarkBlue,%
		linewidth=1pt,topline=true,,%
		frametitleaboveskip=\dimexpr-\ht\strutbox\relax
	}
	\begin{mdframed}[backgroundcolor=cyan!20]\relax%
		\label{#2}}{\vspace{4pt}\end{mdframed}}
	
% for adjustwidth environment
\usepackage[strict]{changepage}

% for formal definitions
\usepackage{framed}

% environment derived from framed.sty: see leftbar environment definition
\definecolor{formalshade}{rgb}{0.95,0.95,1}

%\newenvironment{formal}{%
%	\def\FrameCommand{%
%		\hspace{1pt}%
%		{\color{DarkBlue}\vrule width 2pt}%
%		{\color{cyan!20}\vrule width 4pt}%
%		\colorbox{cyan!20}%
%	}%
%	\MakeFramed{\advance\hsize-\width\FrameRestore}%
%	\noindent\hspace{-4.55pt}% disable indenting first paragraph
%	\begin{adjustwidth}{}{7pt}%
%		\vspace{2pt}\vspace{2pt}%
%	}
%	{%
%		\vspace{2pt}\end{adjustwidth}\endMakeFramed%
%}

\interfootnotelinepenalty=10000 % Keeps footnotes on one page

\newenvironment{formal}{%
	\def\FrameCommand{%
	{}%
	{}%
}%
\MakeFramed{\begin{mdframed}[backgroundcolor=cyan!20, bottomline=false, topline=false, rightline=false, linewidth=2pt, linecolor=DarkBlue]\relax\advance\hsize-\width\FrameRestore}%
\noindent\hspace{-4.55pt}% disable indenting first paragraph
\begin{adjustwidth}{}{7pt}%
}
{%
	\end{adjustwidth}\vspace{2pt}\end{mdframed}\endMakeFramed%
}

\usepackage{tikz,lipsum,lmodern}
\usepackage[most]{tcolorbox}

\newcommand{\imagebox}[2]{
	\begin{tcolorbox}[enhanced,frame style image=../images/blue-abstract.jpg, 
		opacityback=0.75,opacitybacktitle=0.25,
		colback=blue!5!white,colframe=blue!75!black,title=#1]
		#2
	\end{tcolorbox}
}


\newcommand\blankfootnote[1]{%
	\let\thefootnote\relax\footnotetext{#1}%
	\let\thefootnote\svthefootnote%
}



\AtBeginBibliography{\small} % Makes the references smaller.